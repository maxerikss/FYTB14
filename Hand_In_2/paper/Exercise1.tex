\section{First Exercise}
\paragraph{Problem:} Three identical springs are connected in a row and then attached between a pair of walls. Two identical masses are attached at the connection points between the central spring and the two others. Using Lagrangian Mechanics, determine the frequencies for horizontal longitudinal vibrations.

\paragraph{Solution:} Assuming the springs are massless and of length $l$ with a spring constant $k$, we can describe the system by setting the coordinate for the left mass to $x_1$ and the right mass $x_2$. Then by putting the origin at the left wall and the the distance $L$ between the walls we obtain the figure below.

\begin{figure}[H]
    \centering
    \input{figures/problem.pdf_tex}
    \caption{Problem setup}
\end{figure}

The kinetic energy of the system is the sum of the kinetic energy of the two masses.
\begin{equation*}
    T = \frac{1}{2} m ( \dot{x}_1^2 + \dot{x}_2^2 )
\end{equation*}
The potential is the sum of the potentials for the different springs
\begin{align*}
    V_\mathrm{left} &= \frac{1}{2} k (x_1 - l)^2\\
    V_\mathrm{middle} &= \frac{1}{2} k (x_2 - x_1 - l)^2\\
    V_\mathrm{right} &= \frac{1}{2} k (L - x_1 - l)^2
\end{align*}
Thus the Lagrangian is 
\begin{equation*}
    \mathcal{L} = T - V_\mathrm{left} - V_\mathrm{middle} -V_\mathrm{right} = \frac{1}{2} m ( \dot{x}_1^2 + \dot{x}_2^2 ) - \frac{1}{2} k [ (x_1 - l)^2 + (x_2 - x_1 - l)^2 + (L - x_2 - k)^2  ]
\end{equation*}

Setting up the Euler-Lagrange equation 
\begin{equation*}
    \frac{\partial \mathcal{L}}{\partial x_i} = \frac{\dd}{\dd t} \left( \frac{\partial \mathcal{L}}{\partial \dot{x}_i} \right) \quad \text{for } i = 1, 2
\end{equation*}
we obtain the system of equations
\begin{equation*}
    \left\{\begin{array}{@{}>{\displaystyle}l@{}}
        k (2 x_1 - x_2) = m \ddot{x}_1\\
        k(2 x_2 - x_1 - L) = m \ddot{x}_2
    \end{array}\right.
\end{equation*}
The equilibrium for the masses occur when $\ddot{x}_1 = \ddot{x}_2 = 0$ and since $k \neq 0$ we get the equilibrium points $x_1^0$ and $x_2^0$ when
\begin{equation*}
    \left\{\begin{array}{@{}>{\displaystyle}l@{}}
        2 x_1^0 - x_2^0 = 0\\
        2 x_2^0 - x_1^0 - L = 0
    \end{array}\right.
    \iff 
    \left\{\begin{array}{@{}>{\displaystyle}l@{}}
        x_2^0 = \frac{2L}{3}\\[8pt]
        3x_1^0 = \frac{L}{3}
    \end{array}\right.
\end{equation*}
Doing a change of coordinates
\begin{equation*}
    \left\{\begin{array}{@{}>{\displaystyle}l@{}}
        x_1 \to u_1 = x_1 - x_1^0 = x_1 - \frac{L}{3}\\[8pt]
        x_2 \to u_2 = x_2 - x_2^0 = x_2 - \frac{2L}{3}
    \end{array}\right. 
    \iff
    \left\{\begin{array}{@{}>{\displaystyle}l@{}}
        \ddot{x}_1 = \ddot{u}_1 \\[8pt]
        \ddot{x}_2 = \ddot{u}_2 
    \end{array}\right. 
\end{equation*}
gives us the system
\begin{equation*}
    \left\{\begin{array}{@{}>{\displaystyle}l@{}}
        k (2 u_1 - u_2) = m \ddot{u}_1\\
        k(2 u_2 - u_1) = m \ddot{u}_2
    \end{array}\right.
    \iff
    \left\{\begin{array}{@{}>{\displaystyle}l@{}}
        \ddot{u}_1 = \frac{k}{m} (2 u_1 - u_2) \\[8pt]
        \ddot{u}_2 = \frac{k}{m} (2 u_2 - u_1)
    \end{array}\right.
\end{equation*}
Doing the ansatz $u_i = C_i e^{i\omega t}$ we can write the system as a matrix equation
\begin{equation*}
    -\omega^2 \begin{bmatrix}
        C_1\\
        C_2
    \end{bmatrix}
    = \frac{k}{m}
    \begin{bmatrix}
        2 & -1\\
        -1 & 2
    \end{bmatrix}
    \begin{bmatrix}
        C_1\\
        C_2
    \end{bmatrix}
    \iff
    \left( 
    \frac{k}{m}
    \begin{bmatrix}
        2 & -1\\
        -1 & 2
    \end{bmatrix}
    +
    \omega^2
    \mathbb{I}
    \right) 
    \begin{bmatrix}
        C_1\\
        C_2
    \end{bmatrix} = 0.
\end{equation*} 
It then follows that 
\begin{equation*}
    0 =
    \begin{vmatrix}
    2k/m + \omega^2 & -k/m\\
    -k/m & 2k/m + \omega^2
    \end{vmatrix}
    = \left(\frac{2k}{m} + \lambda\right)^2 - \frac{k^2}{m^2} = \omega^4 -\frac{4k}{m} \omega^2 + \frac{3k^2}{m^2}.
\end{equation*}
The pq-formula then gives 
\begin{equation*}
    \omega^2 = \frac{2k}{m} \pm \frac{k}{m} \iff \omega_1 = \sqrt{\frac{k}{m}},\: \omega_2 = \sqrt{\frac{3k}{m}}
\end{equation*}

\paragraph{Answer:} The vibrational frequencies are 
\begin{align*}
    \omega_1 &= \sqrt{\frac{k}{m}}\\[8pt]
    \omega_2 &= \sqrt{\frac{3k}{m}}
\end{align*}