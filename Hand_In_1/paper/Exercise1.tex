\section{Sliding Mass on Sliding Sled}

\paragraph{Problem:} A sled of mass $M$ slides down a hill with a constant slope $\alpha$ relative to the horizontal. The sled is wedge-shaped and its upper edge forms the angle $\beta$ relative to the slope of the hill. On top of the sled is a box of mass m that can slide down the the sled surface. None of the parts experience any friction. Find convenient coordinates and write the Lagrange function of the system, then write down all Euler-Lagrange equations, and solve the equations under the following assumptions: The whole system starts at rest, the sled is not yet at the foot of the hill, and the box is not yet at the foot of the sled.

\begin{figure}[H]
    \centering
    \input{figures/problem.pdf_tex}
    \caption{Problem setup}
    \label{fig:problem}
\end{figure}




\paragraph{Solution:} We can solve this problem using a Lagrangian $\mathcal{L} = T - V$ where $T$ is the kinetic energy of the system and $V$ is the potential energy of the system. Writing $T = T_1 + T_2$, where $T_1$ is the kinetic energy of the sled and $T_2$ is the kinetic energy of the box. Similarly, we write $V = V_1 + V_2$ where $V_1$ is the potential energy of the sled and $V_2$ is the potential energy of the box. Using the coordinates $q_1$ parallel to the hill, that is it is tilted clockwise by $\alpha$ from the horizontal axis and $q_2$ parallel to the sled, that is it is tilted clockwise by $\alpha + \beta$ from the horizontal axis and setting the origin at the centre of mass, and the mass $M$ for the sled and $m$ for the box, we can define the energies. Refer to Fig. \ref*{fig:problem}.

Starting with the kinetic energies, it is simple to see that $T_1 = \frac{1}{2} M \dot{q}_1^2$. However, the coordinate for the box is moving with the sled, thus we need to calculate the components separately. Let the velocity of the box be $v$, then the component parallel to $q_1$ can be described by $v_\parallel = \dot{q}_1 + \dot{q}_2 \cos \beta$. The component perpendicular to $q_1$ is described by $v_\perp = \dot{q}_2 \sin \beta$. Since $v^2 = v_\parallel^2 + v_\perp^2$ we can perform the following calculation
\begin{align*}
    v^2 &= v_\parallel^2 + v_\perp^2 = (\dot{q}_1 + \dot{q}_2 \cos \beta)^2 + (\dot{q}_2 \sin \beta)^2 = \dot{q}_1^2 + 2 \dot{q}_1 \dot{q}_2 \cos \beta + \dot{q}_2^2 \cos^2 \beta + \dot{q}_2^2 \sin^2 \beta\\
    &= \dot{q}_1^2 + 2 \dot{q}_1 \dot{q}_2 \cos \beta + \dot{q}_2^2 (\cos^2 \beta + \sin^2 \beta) = \dot{q}_1^2 + 2 \dot{q}_1 \dot{q}_2 \cos \beta + \dot{q}_2^2.
\end{align*}
We can now write $T_2 = \frac{1}{2} m v^2 = \frac{1}{2} m (\dot{q}_1^2 + 2 \dot{q}_1 \dot{q}_2 \cos \beta + \dot{q}_2^2)$.

For the potential energies the energy for the sled is also simpler to calculate. By setting the potential energy to zero at $t=0$, we only need to calculate the vertical distance traveled down, and thus $V_1 = - M g q_1 \sin \alpha$ where $g$ is the gravitational acceleration. The potential energy for the box is a little trickier. First we set the potential energy to zero at the starting position, then we need to find the vertical distance to the hill which is $q_2 \sin (\alpha + \beta)$ and then the vertical distance from the sled to the ground which is as before $q_1 \sin \alpha$. Thus the potential energy is $V_2 = - m g (q_1 \sin \alpha + q_2 \sin(\alpha + \beta))$.

We can now write the Lagrangian
\begin{equation}
    \mathcal{L} = \frac{1}{2} M \dot{q}_1^2 + \frac{1}{2} m (\dot{q}_1^2 + 2 \dot{q}_1 \dot{q}_2 \cos \beta + \dot{q}_2^2) + M g q_1 \sin \alpha + m g (q_1 \sin \alpha + q_2 \sin(\alpha + \beta)).
\end{equation}
Using the Euler-Lagrange equation
\begin{equation}
    \frac{\partial \mathcal{L}}{\partial q_i} = \frac{\dd }{\dd t} \left( \frac{\partial \mathcal{L}}{\partial \dot{q_i}} \right) \quad \text{for } i = 1, 2
\end{equation}
we get a system of equations
\begin{equation}
    \left.
        \begin{array}{@{}>{\displaystyle}l@{}}
            i = 1\\
            i = 2
        \end{array}
    \right.
    \left\{
        \begin{array}{@{}>{\displaystyle}l@{}}
            (M + m) g \sin \alpha = M \ddot{q}_1 + m (\ddot{q}_1 + \ddot{q}_2 \cos \beta)\\
            g \sin (\alpha + \beta) = \ddot{q}_1 \cos\beta + \ddot{q}_2
        \end{array}\label{eq:LagrangeSystem}
    \right..
\end{equation}
Solving the bottom equation for $\ddot{q}_2$ we obtain
\begin{equation}
    \ddot{q}_2 = g \sin (\alpha + \beta) - \ddot{q}_1 \cos \beta.\label{eq:q2ddot}
\end{equation}
Substituting in Eq. \eqref{eq:q2ddot} in the top equation of Eq. \eqref{eq:LagrangeSystem} we obtain
\begin{align}
    (M + m) g \sin \alpha &= M \ddot{q}_1 + m \ddot{q}_1 + m \cos \beta (g \sin (\alpha + \beta) - \ddot{q}_1 \cos\beta)\nonumber\\
    (M + m) g \sin \alpha &= \ddot{q}_1 (M + m - m \cos^2 \beta) + m g \sin(\alpha + \beta) \cos\beta \nonumber\\
    \ddot{q}_1 &= \frac{(M + m) g \sin \alpha - m g \sin(\alpha + \beta) \cos\beta}{M + m \sin^2 \beta}. \label{eq:q1ddot}
\end{align}
Substituting in Eq. \eqref{eq:q1ddot} into Eq. \eqref{eq:q2ddot} we get
\begin{align*}
    \ddot{q}_2 &= g \sin (\alpha + \beta) - \frac{(M + m) g \sin \alpha \cos \beta - m g \sin(\alpha + \beta) \cos^2 \beta}{M + m \sin^2 \beta}\\
    \ddot{q}_2 &= \frac
    {
        (M + m \sin^2 \beta) g \sin (\alpha + \beta) -
        (M + m) g \sin \alpha \cos \beta +
        m g \sin(\alpha + \beta) \cos^2\beta
    }
    {M + m \sin^2 \beta}\\
    \ddot{q}_2 &= \frac
    {
        g \sin (\alpha + \beta) (M + m\sin^2 \beta + m \cos^2 \beta) -
        (M + m) g \sin\alpha \cos\beta
    }
    {M + m \sin^2 \beta}\\
    \ddot{q}_2 &= \frac
    {
        g \sin (\alpha + \beta) (M + m) -
        (M + m) g \sin\alpha \cos\beta
    }
    {M + m \sin^2 \beta}\\
    \ddot{q}_2 &= \frac
    {
        g(M + m)
        (\sin (\alpha + \beta) -
        \sin\alpha \cos\beta)
    }
    {M + m \sin^2 \beta}\\
    \ddot{q}_2 &= \frac
    {
        g(M + m) \sin\beta \cos\alpha
    }
    {M + m \sin^2 \beta}.
\end{align*}
Integrating $\ddot{q}_1$ and $\ddot{q}_2$ twice, we get 
\begin{align}
    q_1 &= \int\left(\int \ddot{q_1} \dd t\right) \dd t = \frac{(M + m)  \sin \alpha - m  \sin(\alpha + \beta) \cos\beta}{M + m \sin^2 \beta} \times \frac{g t^2}{2} + t v_1^0 + q_1^0\\
    q_2 &= \int\left(\int \ddot{q_2} \dd t\right) \dd t =\frac{(M + m) \sin\beta \cos\alpha }{M + m \sin^2 \beta} \times \frac{g t^2}{2} + t v_2^0 + q_2^0.
\end{align}
Since the system is at rest initially $v_1^0 = v_2^0 = 0$.

\paragraph{Answer:} The solution of the system is 
\begin{align*}
    q_1 &= \frac{(M + m)  \sin \alpha - m  \sin(\alpha + \beta) \cos\beta}{M + m \sin^2 \beta} \times \frac{g t^2}{2} + q_1^0\\
    q_2 &= \frac{(M + m) \sin\beta \cos\alpha }{M + m \sin^2 \beta} \times \frac{g t^2}{2} + q_2^0.
\end{align*}